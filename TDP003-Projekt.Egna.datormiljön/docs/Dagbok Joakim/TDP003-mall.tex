\documentclass{TDP003mall}



\newcommand{\version}{Version 1.1}
\author{Joakim Johansson, \url{joajo229@student.liu.se}}
\title{DAGBOK}
\date{2016-09-14}
\rhead{Joakim Johansson}


\begin{document}
\projectpage

\section*{2016-09-14}
Vi satte upp git så att allt ska fungera när vi väl börjar med projektet.

\section*{2016-09-15}
Idag har vi skapat alla filer med början på funktionerna osv enligt kravspecen. Så nu har vi koll på allt men strukturen på filerna i mappen osv

\section*{2016-09-16}
Lagt till en .gitignore fil så att inte onödiga filer ska synca när man pushar och pullar. Det är bra att ha med *~ i .gitignore filen så att man slipper alla temp filer som emac skapar.

Vi lade även till alla standard funktionerna som man var tvungen att ha med i Datalagret, tillsammans med commentarer som beskriver vad varje funktion ska göra och dess ut och indata. Vi började även på API:et lite gran med en funktion.

\section*{2016-09-19}
Nu är nästan hela API:et klart, bara Search funktionens kvar som vi tror kommer vara den jobbigaste...

Idag har vi även jobbat med Användarmanualen och lagt till bilder i ett latex dokument tillsammans med förklarande text så att man har beskrivit hur användaren ska göra i vissa situvationer och vad som händer när man klickar på olika ställen.

\section*{2016-09-22}
Idag har vi jobbat med projektplaneringen och gjort ett grovt första kast med överskrifter som är hyfsat relevanta. Vi ska senare kolla på kraven och se vad de vill att det ska finnas med i en projektplan.

\section*{2016-09-23}
Idag har jag skapat min dagbok och försökt att minnas vad som jag gjort alla de andra dagar innan denna dag. Men genom att kolla på git commitsen så var det inte så svårt att komma ihåg vad man gjort. Det är ju nästan som en dagbok. Man ser i alla fall vilka saker man har gjort när men sen gäller det ju att komma ihåg mer om varje dag så att man kan skriva det i dagboken. Förhoppningsvis blir det bättre dagboksinlägg från och med nu!

\section*{2016-09-25}
Idag har jag jobbat med Datalagret och Flask. Har satt upp en enkel grupp med funktioner som täcker ungefär alla funktioner som ska behövas ha med i den slutgiltiga koden. Har även skrivit lite kod så att det ska bli lätt att koppla ihop det med API:et senare sen.

\section*{2016-09-26}
Arturas jobbade med Datalagret i helgen och vi slutstälde det idag, med Search funktioner och alla de andra också, med OK på alla tester med testfilen. Vi ändrade sedan även lite i flask-koden så att det ska fungera ihop med API:et, men Flask-koden är långt ifrån klart.

Vi fick idag komplitering på projektplanen och har jobbat lite med den och lagt in tabeller så att man lättare ska se vad som händer i listorna.

\section*{2016-09-27}
Lagt till bättre verioner av Lofi-Prototyperna och sammlat dem i mappen /doc/LoFi

\section*{2016-09-28}
Denna dag så har vi jobbat med att få till projektplanen så att den blir fin och går igenom alla krav.

Vi fixade även lite mer saker inne i flask filen så att den ska kunna hantera mer sidor.

\section*{2016-10-04}
Idag har jag jobbat med HTML och CSS för att bli klar med en första draft för hemsidans utseende.

\section*{2016-10-05}
Idag har vi skapat de andra html sidorna, men de är inte klara ännu, sedan så har vi även jobbat på data.py för att fixa så att de kan hantera de nya sidorna på ett bra sätt med information osv.

\section*{2016-10-06}
Idag gjorde vi nästan klart listsidan så att den fungerar med sökning och filtrering av projekten. Vi fixade även till en bugg som fanns i datalagret/API:et då den inte kunde hantera en viss data ifrån söksidan.

\section*{2016-10-11}
Denna dag har vi gjort väldigt mycket. Nu fungerar alla sidorna och vi har även lagt till kommentarer på engelska i alla filer.

\section*{2016-10-12}
Denna dag har vi börjat med systemdokumentationen och lagt in en massa överskrifter och exempeltext över allt för att få en översikt över vad som kommer vara med i dokumentet. Vi började även skriva lite på de tre första överskrifterna.

\section*{2016-10-13}
Idag har vi skrivit på systemdokumentationen båda två, och slått ihop dom med jämna mellanrum, supersmidigt med git mearge!

Första utkastet av systemdokumentationen är nu klar och inskickat.



\end{document}
